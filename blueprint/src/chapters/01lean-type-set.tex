\chapter{Leanの型理論と集合論}

\section{型理論}
Leanの基礎付けは型理論(type theory)であって,集合論(set theory)ではない.
従って,集合論とは微妙に異なる振る舞いをすることがある.
以下では2つの統語的な違いを説明する.

% ## Brief note on type theory

% Lean is based on type theory,
% which means that some things work slightly differently from set theory.
% We highlight two syntactical differences.

\begin{itemize}
    \item まず,型理論において包含関係($\in$, element-of relation)は基本的な関係ではない.その代わりに,型判断($\,:\,$, typying judgement)がある.
    % Firstly, the element-of relation (`∈`) plays no fundamental role.
    % Instead, there is a typing judgment (`:`).
    このため,「$x$は集合$X$の要素である」と言う代わりに,「$x:X$」と書いて「$x$は型$X$の項である」と言う.
    都合の良いことに,「$f:X \to Y$」と書いて「$f$の型は$X \to Y$である」と言える.そしてこれは「$f$が$X$から$Y$への写像である」ことを意味する.
    % This means that we write `x : X` to say that "`x` is a term of type `X`"
    % instead of "`x` is an element of the set `X`".
    % Conveniently, we can write `f : X → Y` to mean "`f` has type `X → Y`",
    % in other words "`f` is a function from `X` to `Y`".
\end{itemize}

  \begin{itemize}
    \item もう一つ,型理論では$\mapsto$よりも$\lambda$記法(lambda-notation)を用いる.
    例えば整数の自乗関数は$\lambda x. x^2$と定義される.これは集合論における$x \mapsto x^2$に相当する.
    % Secondly, type theorists do not use the mapsto symbol (`↦`),
    % but instead use lambda-notation.
    % This means that we can define the square function on the integers via
    % `λ x, x^2`, which translates to `x ↦ x^2` in set-theoretic notation.
    % For more information about `λ`, see the Wikipedia page on
    % [lambda calculus](https://en.wikipedia.org/wiki/Lambda_calculus).
  \end{itemize}

  Lean の型理論については,
% For a more extensive discussion of type theory,
% see the dedicated
\begin{itemize}
  \item \url{https://leanprover-community.github.io/lean-perfectoid-spaces/type_theory.html}
\end{itemize}
を見よ.

% on the perfectoid project website.

\section{Leanの集合論}
\begin{itemize}
  \item \texttt{Mathlib.Data.Set.Defs}
\end{itemize}

Leanにおいて,
型$\alpha$の項からなる集合 $a : \texttt{Set } \alpha$ は,述語$\alpha \to \mathtt{Prop}$として実装されている.
これは実装上の都合であって,この実装を当てにするべきではない(de Moura).
その代わり,述語 $p : \alpha \to \mathtt{Prop}$ を集合$\{x \mid p x\}$に変換する関数 $\texttt{setOf } p : \texttt{Set } \alpha := p$ と,
包含関係 $a \in p$ に変換する関数 $\texttt{Mem } (s : p) (a : \alpha) : \mathtt{Prop} := s a$ を使用することが推奨されている.

\begin{itemize}
    \item \texttt{Mathlib.Data.Finset}
\end{itemize}

Mathlibにはまた,有限集合が定義されている.有限集合では帰納法が使える.
